%% \documentclass[suppldata]{gOMS2e}   % suppldata suppresses "To appear in ...." info"
\documentclass[10pt,reqno]{amsart}
\usepackage[left=.5in,right=.5in,top=.5in,bottom=.5in]{geometry}


\usepackage{amssymb}
\usepackage{subfigure}% Support for small, `sub' figures and tables
\usepackage{booktabs}
\usepackage{tikz}

\usepackage{verbatim}

\usepackage{tikz}
%%%% Use png graphics, some pdf graphics:
%\graphicspath{{graphics-for-algo-paper/}{graphics-for-algo-paper/animation-graphics/}}
%%%% Use pdf graphics:
%%\graphicspath{{graphics-for-algo-paper/software-paper-graphics-pdf/}{graphics-for-algo-paper/animation-graphics/}}

\definecolor{mediumspringgreen}{rgb}{0.0, 0.98039215, 0.60392156}

\newcommand{\bb}{\mathbb}
\newcommand{\R}{\bb R}
\newcommand{\Q}{\bb Q}
\newcommand{\Z}{\bb Z}
\newcommand{\N}{\bb N}

% No longer mathcal
\newcommand{\B}{B}
\renewcommand{\P}{\mathcal{P}}

\newcommand\st{:}
\newcommand{\setcond}[2]{\left\{ #1 \,:\, #2 \right\}}

\DeclareMathOperator    \relint         {rel\,int}
\DeclareMathOperator    \verts          {vert}
\DeclareMathOperator    \intr                   {int}
\DeclareMathOperator    \cl                     {cl}


\chardef\Myunderscore=`\_
%\let\Myunderscore=\textunderscore
% from http://texwelt.de/wissen/fragen/565/was-heit-hyperrefs-warnung-token-not-allowed-in-a-pdf-string
\usepackage{hyperref}  
\pdfstringdefDisableCommands{%
  \def\Myunderscore{\textunderscore}%
}
\newcommand\underscore{\Myunderscore\allowbreak}
\let\_=\underscore

\newcommand\githubsearchurl{https://github.com/mkoeppe/infinite-group-relaxation-code/search}
\usepackage{pgf}
\DeclareRobustCommand\sage[1]{\texttt{#1}}


 \newcommand{\tred}[1]{\texttt{\textcolor {red} {#1}}}

\makeatletter %% Fix for spacing of three stars
\def\@fnsymbol#1{\ensuremath{\ifcase#1\or\star\or{\star\star}\or
   {\star{\star}\star}\or \dagger\or \ddagger\or
   \mathchar "278\or \mathchar "27B\or \|\or **\or \dagger\dagger
   \or \ddagger\ddagger \else\@ctrerr\fi}}
\makeatother


\begin{document}

\setcounter{page}{12}
\setcounter{section}{4}

\section{Figures from \itshape Equivariant Perturbation in Gomory and Johnson's Infinite Group 
  Problem. V. Software for the continuous and discontinuous 1-row case}

\begin{figure}[h]
\centering
\begin{minipage}{.49\textwidth}
\centering
\includegraphics[width=.8\linewidth]{gmic_plot}
\end{minipage}
\begin{minipage}{.49\textwidth}
\centering
\includegraphics[width=.8\linewidth]{discont_pwl_plot}
\end{minipage}
\caption{Two piecewise linear functions, as plotted by the command
  \sage{plot\_with\_colored\_slopes(h)}. \textit{Left,} continuous extreme
  function \sage{h = gmic()}. \textit{Right,} random discontinuous function
  \sage{h = equiv5\_random\_discont\_1()}, generated by \sage{random\_piecewise\_function(xgrid=5, ygrid=5, continuous\_proba=1/3, symmetry=True)}.}
\label{fig:cont_and_discont_pwl_functions}
\end{figure}

\begin{figure}[h]
  \centering
  \includegraphics[width=.5\linewidth]{construct_a_face}
  \caption{An example of a face $F = F(I, J, K)$ of the 2-dimensional
    polyhedral complex $\Delta\P$, set up by \sage{F = Face([[0.2, 0.3],
      [0.75, 0.85], [1, 1.2]])}.  It has vertices (\emph{blue})
    $(0.2, 0.85)$, $(0.3, 0.75)$, $(0.3, 0.85)$, $(0.2, 0.8)$, $(0.25, 0.75)$,
    whereas the other basic solutions (\emph{red})
    $(0.2, 0.75)$, $(0.2, 1)$, $(0.3, 0.9)$, $(0.35, 0.85)$, $(0.45, 0.75)$
    are filtered out because they are infeasible. 
    The face $F$ has projections (\emph{gray shadows})
    $I' = p_1(F) = [0.2, 0.3]$ (\emph{top border}), $J' = p_2(F) = [0.75,
    0.85]$ (\emph{left border}), and $K'
      = p_3(F) = [1, 1.15]$ (\emph{right border}). Note that $K'\subsetneq K$. 
  } 
  \label{fig:construct_a_face}
\end{figure}

\begin{figure}[h]
\centering
\begin{minipage}{.49\textwidth}
\centering
\includegraphics[width=.9\linewidth]{2d_diagram_with_cones_continuous_deco}
\end{minipage}
\begin{minipage}{.49\textwidth}
\centering
\includegraphics[width=.9\linewidth]{2d_diagram_with_cones_discontinuous}
\end{minipage}
\caption{Two diagrams of functions and their polyhedral complexes $\Delta\P$ with colored cones at $\verts(\Delta\P)$, as plotted by the command \sage{plot\_2d\_diagram\_with\_cones(h)}. \textit{Left}, continuous function \sage{h = not\_minimal\_2()}. \textit{Right}, random discontinuous function \sage{h = equiv5\_random\_discont\_1()}.}
\label{fig:2d_diagram_with_cones}
\end{figure}

  \begin{figure}[h]
    \centering
    \begin{minipage}{.49\textwidth}
      \centering
      \includegraphics[width=.9\linewidth]{2d_with_cones_example7slopecoarse2}
    \end{minipage}
    \begin{minipage}{.49\textwidth}
      \centering
      \includegraphics[width=.9\linewidth]{2d_with_faces_example7slopecoarse2}
    \end{minipage}
    \caption{Diagrams of $\Delta\P$ of a continuous function \sage{h
        = example7slopecoarse2()}, with (\textit{left}) additive vertices as
      plotted by the command \sage{plot\_2d\_diagram\_with\_cones(h)};
      (\textit{right}) maximal additive faces as plotted by the command
      \sage{plot\_2d\_diagram(h)}.}
    \label{fig:2d_diagrams_continuous_function}
  \end{figure}


\begin{figure}[h]
\centering
\begin{minipage}{.49\textwidth}
\centering
\includegraphics[width=.9\linewidth]{2d_with_cones_discontinuous}
\end{minipage}
\begin{minipage}{.49\textwidth}
\centering
\includegraphics[width=.9\linewidth]{2d_with_faces_discontinuous}
\end{minipage}
\caption{Diagrams of $\Delta\P$ of a discontinuous function \sage{h = hildebrand\underscore discont\underscore 3\underscore slope\underscore 1()}, with (\textit{left}) additive limiting cones as plotted by the command \sage{plot\_2d\_diagram\_with\_cones(h)}; (\textit{right}) additive faces as plotted by the command \sage{plot\_2d\_diagram(h)}.}
\label{fig:2d_diagrams_discontinuous_function}
\end{figure}


\begin{figure}[h]
\centering
\includegraphics[width=.23\linewidth]{gj2slope-0}
\includegraphics[width=.23\linewidth]{gj2slope-1}
\includegraphics[width=.23\linewidth]{gj2slope-2}
\includegraphics[width=.23\linewidth]{gj2slope-3}
\caption{Compute the (directly) covered intervals for \sage{$\pi =$
    gj\_2\_slope(3/5,1/3)}.
  \textcolor{red}{Note \sage{generate\_animation\_2d\_diagram} has been replaced by
    \sage{generate\_covered\_components\_strategically}, which does not
    indicate components by color.}
}
\label{fig:compute_covered_intervals_cont}
\end{figure}

\clearpage
\begin{figure}[h]
\centering
\includegraphics[width=.32\linewidth]{animation_2d_diagram_disc-0}
\includegraphics[width=.32\linewidth]{animation_2d_diagram_disc-1}
\includegraphics[width=.32\linewidth]{animation_2d_diagram_disc-2}
\\
\includegraphics[width=.32\linewidth]{animation_2d_diagram_disc-3}
\includegraphics[width=.32\linewidth]{animation_2d_diagram_disc-4}
\includegraphics[width=.32\linewidth]{animation_2d_diagram_disc-5}
\caption{Compute the (directly and indirectly) covered intervals for
  \sage{$\pi=$ hildebrand\_discont\_3\_slope\_1()}.  \textcolor{red}{Note
    \sage{generate\_animation\_2d\_diagram} has been replaced by
    \sage{generate\_covered\_components\_strategically}, which does not
    indicate components by color.}  }
\label{fig:compute_covered_intervals_disc}
\end{figure} 

\setcounter{table}{2}
\begin{table}[h]
  \caption{A sample Sage session on the extremality test}
  \label{tab:extremality_test}
  {\includegraphics[scale=0.5]{not-extreme-perturbation-1}}
\end{table}

\begin{table}[h]
  \caption{A sample Sage session on discrete functions for the finite group problem.}
  \label{tab:discrete-funtions}
  \includegraphics[scale=0.6]{gmic45}
  \includegraphics[scale=0.6]{gmic45-restricted}
  \includegraphics[scale=0.6]{gmic45-auto-rest}
  \includegraphics[scale=0.6]{gmic45-auto-interpol}
\end{table}
%\clearpage


\end{document}

%%% Local Variables:
%%% mode: latex
%%% TeX-master: t
%%% End:
