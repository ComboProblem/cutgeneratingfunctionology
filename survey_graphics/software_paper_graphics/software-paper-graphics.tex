%% \documentclass[suppldata]{gOMS2e}   % suppldata suppresses "To appear in ...." info"
\documentclass[10pt,reqno]{amsart}
\usepackage[left=.5in,right=.5in,top=.5in,bottom=.5in]{geometry}

%%%% Use png graphics, some pdf graphics:
%\graphicspath{{graphics-for-algo-paper/}{graphics-for-algo-paper/animation-graphics/}}
%%%% Use pdf graphics:
\graphicspath{{graphics-for-algo-paper/software-paper-graphics-pdf/}{graphics-for-algo-paper/animation-graphics/}}

\usepackage{amssymb}
\usepackage{subfigure}% Support for small, `sub' figures and tables
\usepackage{booktabs}
\usepackage{tikz}

\usepackage{verbatim}

\usepackage{tikz}
\definecolor{mediumspringgreen}{rgb}{0.0, 0.98039215, 0.60392156}

\newcommand{\bb}{\mathbb}
\newcommand{\R}{\bb R}
\newcommand{\Q}{\bb Q}
\newcommand{\Z}{\bb Z}
\newcommand{\N}{\bb N}

% No longer mathcal
\newcommand{\B}{B}
\renewcommand{\P}{\mathcal{P}}

\newcommand\st{:}
\newcommand{\setcond}[2]{\left\{ #1 \,:\, #2 \right\}}

\DeclareMathOperator    \relint         {rel\,int}
\DeclareMathOperator    \verts          {vert}
\DeclareMathOperator    \intr                   {int}
\DeclareMathOperator    \cl                     {cl}


\chardef\Myunderscore=`\_
%\let\Myunderscore=\textunderscore
% from http://texwelt.de/wissen/fragen/565/was-heit-hyperrefs-warnung-token-not-allowed-in-a-pdf-string
\usepackage{hyperref}  
\pdfstringdefDisableCommands{%
  \def\Myunderscore{\textunderscore}%
}
\newcommand\underscore{\Myunderscore\allowbreak}
\let\_=\underscore

\newcommand\githubsearchurl{https://github.com/mkoeppe/infinite-group-relaxation-code/search}
\usepackage{pgf}
\DeclareRobustCommand\sage[1]{\texttt{#1}}


 \newcommand{\tred}[1]{\texttt{\textcolor {red} {#1}}}

\makeatletter %% Fix for spacing of three stars
\def\@fnsymbol#1{\ensuremath{\ifcase#1\or\star\or{\star\star}\or
   {\star{\star}\star}\or \dagger\or \ddagger\or
   \mathchar "278\or \mathchar "27B\or \|\or **\or \dagger\dagger
   \or \ddagger\ddagger \else\@ctrerr\fi}}
\makeatother


\begin{document}

\setcounter{page}{12}
\setcounter{section}{4}

\section{Figures from \itshape Equivariant Perturbation in Gomory and Johnson's Infinite Group 
  Problem. V. Software for the continuous and discontinuous 1-row case}

\begin{figure}[h]
\centering
\begin{minipage}{.49\textwidth}
\centering
\includegraphics[width=.8\linewidth]{gmic_plot}
\end{minipage}
\begin{minipage}{.49\textwidth}
\centering
\includegraphics[width=.8\linewidth]{discont_pwl_plot}
\end{minipage}
\caption{Two piecewise linear functions, as plotted by the command
  \sage{plot\_with\_colored\_slopes(h)}. \textit{Left,} continuous extreme
  function \sage{h = gmic()}. \textit{Right,} random discontinuous function
  \sage{h = equiv5\_random\_discont\_1()}, generated by \sage{random\_piecewise\_function(xgrid=5, ygrid=5, continuous\_proba=1/3, symmetry=True)}.}
\label{fig:cont_and_discont_pwl_functions}
\end{figure}

\begin{figure}[h]
  \centering
  \includegraphics[width=.5\linewidth]{construct_a_face}
  \caption{An example of a face $F = F(I, J, K)$ of the 2-dimensional
    polyhedral complex $\Delta\P$, set up by \sage{F = Face([[0.2, 0.3],
      [0.75, 0.85], [1, 1.2]])}.  It has vertices (\emph{blue})
    $(0.2, 0.85)$, $(0.3, 0.75)$, $(0.3, 0.85)$, $(0.2, 0.8)$, $(0.25, 0.75)$,
    whereas the other basic solutions (\emph{red})
    $(0.2, 0.75)$, $(0.2, 1)$, $(0.3, 0.9)$, $(0.35, 0.85)$, $(0.45, 0.75)$
    are filtered out because they are infeasible. 
    The face $F$ has projections (\emph{gray shadows})
    $I' = p_1(F) = [0.2, 0.3]$ (\emph{top border}), $J' = p_2(F) = [0.75,
    0.85]$ (\emph{left border}), and $K'
      = p_3(F) = [1, 1.15]$ (\emph{right border}). Note that $K'\subsetneq K$. 
  } 
  \label{fig:construct_a_face}
\end{figure}

\begin{figure}[t]
\centering
\begin{minipage}{.49\textwidth}
\centering
\includegraphics[width=.9\linewidth]{2d_diagram_with_cones_continuous_deco}
\end{minipage}
\begin{minipage}{.49\textwidth}
\centering
\includegraphics[width=.9\linewidth]{2d_diagram_with_cones_discontinuous}
\end{minipage}
\caption{Two diagrams of functions and their polyhedral complexes $\Delta\P$ with colored cones at $\verts(\Delta\P)$, as plotted by the command \sage{plot\_2d\_diagram\_with\_cones(h)}. \textit{Left}, continuous function \sage{h = not\_minimal\_2()}. \textit{Right}, random discontinuous function \sage{h = equiv5\_random\_discont\_1()}.}
\label{fig:2d_diagram_with_cones}
\end{figure}

  \begin{figure}[tp]
    \centering
    \begin{minipage}{.49\textwidth}
      \centering
      \includegraphics[width=.9\linewidth]{2d_with_cones_example7slopecoarse2}
    \end{minipage}
    \begin{minipage}{.49\textwidth}
      \centering
      \includegraphics[width=.9\linewidth]{2d_with_faces_example7slopecoarse2}
    \end{minipage}
    \caption{Diagrams of $\Delta\P$ of a continuous function \sage{h
        = example7slopecoarse2()}, with (\textit{left}) additive vertices as
      plotted by the command \sage{plot\_2d\_diagram\_with\_cones(h)};
      (\textit{right}) maximal additive faces as plotted by the command
      \sage{plot\_2d\_diagram(h)}.}
    \label{fig:2d_diagrams_continuous_function}
  \end{figure}


\begin{figure}[tp]
\centering
\begin{minipage}{.49\textwidth}
\centering
\includegraphics[width=.9\linewidth]{2d_with_cones_discontinuous}
\end{minipage}
\begin{minipage}{.49\textwidth}
\centering
\includegraphics[width=.9\linewidth]{2d_with_faces_discontinuous}
\end{minipage}
\caption{Diagrams of $\Delta\P$ of a discontinuous function \sage{h = hildebrand\underscore discont\underscore 3\underscore slope\underscore 1()}, with (\textit{left}) additive limiting cones as plotted by the command \sage{plot\_2d\_diagram\_with\_cones(h)}; (\textit{right}) additive faces as plotted by the command \sage{plot\_2d\_diagram(h)}.}
\label{fig:2d_diagrams_discontinuous_function}
\end{figure}


\begin{figure}[h]
\centering
\includegraphics[width=.32\linewidth]{gj2slope-6}
\includegraphics[width=.32\linewidth]{gj2slope-12}
\includegraphics[width=.32\linewidth]{gj2slope-14}
%\includegraphics[width=.23\linewidth]{gj2slope-15}
\caption{Compute the (directly) covered intervals for \sage{$\pi =$ gj\_2\_slope(3/5,1/3)}.}
\label{fig:compute_covered_intervals_cont}
\end{figure}

\begin{figure}[h]
\centering
\includegraphics[width=.32\linewidth]{animation_2d_diagram_disc-6}
\includegraphics[width=.32\linewidth]{animation_2d_diagram_disc-8}
\includegraphics[width=.32\linewidth]{animation_2d_diagram_disc-10}
\\
\includegraphics[width=.32\linewidth]{animation_2d_diagram_disc-12}
\includegraphics[width=.32\linewidth]{animation_2d_diagram_disc-14}
\includegraphics[width=.32\linewidth]{animation_2d_diagram_disc-15}
\caption{Compute the (directly and indirectly) covered intervals for \sage{$\pi=$  hildebrand\_discont\_3\_slope\_1()}}
\label{fig:compute_covered_intervals_disc}
\end{figure} 

\setcounter{table}{2}
\begin{table}
  \caption{A sample Sage session on the extremality test}
  \label{tab:extremality_test}
  \begin{tikzpicture}[overlay]
     \node at (12.5,-4) {\includegraphics[scale=0.5]{not-extreme-perturbation-1}};
  \end{tikzpicture}
  \tiny %\footnotesize
  \begin{tabular}{@{}p{\linewidth}@{}}
    \toprule
	\begin{verbatim}
	## First we load a function and store it in variable h.
	## We start with the easiest function, the GMIC.
	sage: h = gmic()
	INFO: 2016-08-08 16:51:31,048 Rational case.
	
	## Test its extremality; this will create informative output and plots
	sage: extremality_test(h, show_plots=True)
	INFO: 2016-08-08 16:54:22,014 pi(0) = 0
	INFO: 2016-08-08 16:54:22,016 pi is subadditive.
	INFO: 2016-08-08 16:54:22,016 pi is symmetric.
	INFO: 2016-08-08 16:54:22,018 Thus pi is minimal.
	INFO: 2016-08-08 16:54:22,018 Plotting 2d diagram...
	INFO: 2016-08-08 16:54:22,018 Computing maximal additive faces...
	INFO: 2016-08-08 16:54:22,022 Computing maximal additive faces... done
	Launched png viewer for Graphics object consisting of 25 graphics primitives
	INFO: 2016-08-08 16:54:22,526 Plotting 2d diagram... done
	INFO: 2016-08-08 16:54:22,526 Computing covered intervals...
	INFO: 2016-08-08 16:54:22,527 Computing covered intervals... done
	INFO: 2016-08-08 16:54:22,527 Plotting covered intervals...
	Launched png viewer for Graphics object consisting of 2 graphics primitives
	INFO: 2016-08-08 16:54:22,985 Plotting covered intervals... done
	INFO: 2016-08-08 16:54:22,986 All intervals are covered (or connected-to-covered). 2 components.
	INFO: 2016-08-08 16:54:23,035 Finite dimensional test: Solution space has dimension 0
	INFO: 2016-08-08 16:54:23,035 Thus the function is extreme.
	
	## The docstring tells us that we can set the `f' value using an optional argument.
	sage: h = gmic(1/5)
	INFO: 2016-08-08 16:55:59,440 Rational case.
	## Of course, we know it will still be extreme; but let's test it to
	## see all the informative graphs.
	sage: extremality_test(h, show_plots=True)
	[...]
	True

	## Let's try a different function from the compendium. 
	## We change the parameters a little bit, so that they do NOT satisfy the known
	## sufficient conditions from the literature about this class of functions.
	sage: h = drlm_backward_3_slope(f=1/12, bkpt=4/12)
	INFO: 2016-08-08 17:03:20,438 Conditions for extremality are NOT satisfied.
	INFO: 2016-08-08 17:03:20,439 Rational case.
	
	## Let's run the extremality test.
	sage: extremality_test(h, show_plots=True)
	[...]
	INFO: 2016-08-08 17:03:44,796 Thus pi is minimal.
	[...]
	INFO: 2016-08-08 17:03:46,417 Uncovered intervals: ([[5/12, 2/3]],)
	[...]
	INFO: 2016-08-08 17:03:49,544 Plotting perturbation... done
	INFO: 2016-08-08 17:03:49,545 Thus the function is NOT extreme.
	False
	## Indeed, it's not extreme.  
	## We see a perturbation in magenta and the two perturbed functions in blue and red,
	## whose average is the original function (black).
	
	## Here's the Gomory fractional cut.
	sage: h = gomory_fractional()
	## It is not even minimal:
	sage: minimality_test(h, True)
	INFO: 2016-08-08 17:06:33,647 pi(0) = 0
	INFO: 2016-08-08 17:06:33,648 pi is not minimal because it does not stay in the range of [0, 1].
	False
	
	## There's many more functions to explore. Use the Tab key on the next line to see a collection of those functions.
	sage: extreme_functions.[tab]
	\end{verbatim}
    \\
	\bottomrule
  \end{tabular}
\end{table}

\begin{table}
  \caption{A sample Sage session on discrete functions for the finite group problem.}
  \label{tab:discrete-funtions}
  \tiny %\footnotesize
  \begin{tabular}{@{}p{\linewidth}@{}}
    \toprule
    \begin{tikzpicture}[overlay]
     \node[align=center] at (12.5,-2) {\scriptsize\sage{plot\_with\_colored\_slopes(h)}\\\includegraphics[scale=0.6]{gmic45}};
     \node[align=center]  at (12.5,-6) {\scriptsize \sage{plot\_with\_colored\_slopes(hr)}\\ \includegraphics[scale=0.6]{gmic45-restricted}};
     \node[align=center]  at (12.5,-10) {\scriptsize\sage{plot\_with\_colored\_slopes(ha)}\\ \includegraphics[scale=0.6]{gmic45-auto-rest}};
     \node[align=center]  at (12.5,-14) {\scriptsize \sage{plot\_with\_colored\_slopes(hi)}\\ \includegraphics[scale=0.6]{gmic45-auto-interpol}};
     \end{tikzpicture}
	\begin{verbatim}
	## We load the GMIC function with f=4/5, and store it in variable h.
	sage: h = gmic(f=4/5)
	INFO: 2016-08-08 17:23:05,206 Rational case.
	
	## We can restrict to a finite group problem.
	sage: restrict_to_finite_group?
	Signature:      restrict_to_finite_group(function, f=None, oversampling=None, order=None)
	Docstring:
	Restrict the given function to the cyclic group of given order.
	
	   If order is not given, it defaults to the group generated by the
	   breakpoints of function and f if these data are rational. However,
	   if oversampling is given, it must be a positive integer; then the
	   group generated by the breakpoints of function and f will be
	   refined by that factor.
	
	   If f is not provided, uses the one found by find_f.
	
	   Assume in the following that f and all breakpoints of function lie
	   in the cyclic group and that function is continuous.
	
	   Then the restriction is minimal valid if and only if function is minimal valid. 
	   The restriction is extreme if function is extreme. 
	
	   If, in addition oversampling >= 3, then the following holds: The
	   restriction is extreme if and only if function is extreme. This is
	   Theorem 1.5 in [IR2].
	
	   This oversampling factor of 3 is best possible, as demonstrated by
	   function kzh_2q_example_1 from [KZh2015a].
	   [...]

	sage: hr = restrict_to_finite_group(h)
	INFO: 2016-08-08 17:26:36,047 Rational breakpoints and f; 
	using group generated by them, (1/5)Z
	
	## This function can be set up by providing the breakpoints and the values.
	sage: discrete_function_from_points_and_values(points=[0, 1/5, 2/5, 3/5, 4/5, 1], 
	          values=[0, 1/4, 1/2, 3/4, 1, 0]) == hr
	INFO: 2016-08-8  17:26:37,190 Rational case.
	True

	## The restricted function is extreme for the finite group problem.
	sage: extremality_test(hr)
	INFO: 2016-08-8  17:26:38,121 pi(0) = 0
	INFO: 2016-08-8  17:26:38,123 pi is subadditive.
	INFO: 2016-08-8  17:26:38,123 pi is symmetric.
	INFO: 2016-08-8  17:26:38,124 Thus pi is minimal.
	INFO: 2016-08-8  17:26:38,124 Rational breakpoints and f; 
	using group generated by them, (1/5)Z
	INFO: 2016-08-8  17:26:38,158 Solution space has dimension 0
	INFO: 2016-08-8  17:26:38,158 Thus the function is extreme.
	True
	
	## For the finite group problems, automorphisms are interesting!
	sage: ha = automorphism(hr, 2)
	INFO: 2016-08-08 17:26:41,100 Rational breakpoints and f; 
	using group generated by them, (1/5)Z
	
	## We can interpolate to get a function for the infinite group problem.
	sage: hi = interpolate_to_infinite_group(ha)
	\end{verbatim}
    \\
    \bottomrule
  \end{tabular}
\end{table}
%\clearpage


\end{document}

%%% Local Variables:
%%% mode: latex
%%% TeX-master: t
%%% End:
