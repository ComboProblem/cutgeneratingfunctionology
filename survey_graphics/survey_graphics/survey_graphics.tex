\documentclass[10pt,reqno]{amsart}
\usepackage[left=.5in,right=.5in,top=.5in,bottom=.5in]{geometry}

\usepackage{booktabs,array,ragged2e}
\usepackage{amssymb}
\usepackage{graphicx}
\usepackage{url}
\usepackage[usenames,dvipsnames,svgnames,table]{xcolor}
\usepackage[square,numbers,sort&compress]{natbib}

\usepackage{subfigure}
\usepackage{multirow}

\usepackage{pgf}  % for sagefunc

\def\visible<#1>{}  % beamer command not needed here

\usepackage[utf8]{inputenc}
% \usepackage{xr}

%%\usepackage[english]{babel}
\usepackage{amsfonts}
\usepackage{amsmath}
\usepackage{latexsym}
\usepackage{color}
\usepackage{subfigure}
\usepackage{enumerate}

\usepackage{hyperref}  

%%\graphicspath{{figures/}}

\let\ParaSign=\S

%Math Operators
\DeclareMathOperator    \aff                    {aff}
\DeclareMathOperator    \argmin         {arg\,min}
\DeclareMathOperator    \argmax         {arg\,max}
\DeclareMathOperator    \bd                     {bd}
\DeclareMathOperator    \cl                     {cl}
\DeclareMathOperator    \conv           {conv}
\DeclareMathOperator    \cone           {cone}
\DeclareMathOperator    \dist           {dist}
\DeclareMathOperator    \ep                     {exp}
\DeclareMathOperator    \et                     {ext}
\DeclareMathOperator    \ext                    {ext}
\DeclareMathOperator    \intr                   {int}
\DeclareMathOperator    \lin                    {lin}
\DeclareMathOperator    \proj           {proj}
\DeclareMathOperator    \rec                    {rec}
\DeclareMathOperator    \rk                     {rk}
\DeclareMathOperator    \relint         {rel\,int}
\DeclareMathOperator    \spann          {span}
\DeclareMathOperator    \verts          {vert}
\DeclareMathOperator    \vol                    {vol}
\DeclareMathOperator    \Aff {Aff}  % Grp of invertible affine linear transformations

%Mathbb
\newcommand{\R}{\mathbb R}
\newcommand{\Q}{\mathbb Q}
\newcommand{\Z}{\mathbb Z}
\newcommand{\N}{\mathbb N}
\newcommand{\C}{\mathbb C}
%\newcommand{\T}{\mathbb T}
\renewcommand{\r}{\bar{r}}
%\newcommand{\p}{\bar{p}}

%\newcommand{\verts}{\mathrm{vert}}
%\renewcommand{\intr}{\mathrm{int}}

%Constructed Commands
\newcommand{\floor}[1]{\lfloor#1\rfloor}
\newcommand{\ceil}[1]{\lceil #1 \rceil}
\newcommand\st{\mid}
\newcommand\bigst{\mathrel{\big|}}
\newcommand\Bigst{\mathrel{\Big|}}
\newcommand\biggst{\mathrel{\bigg|}}

%% Vectors
\def\ve#1{\mathchoice{\mbox{\boldmath$\displaystyle\bf#1$}}
{\mbox{\boldmath$\textstyle\bf#1$}}
{\mbox{\boldmath$\scriptstyle\bf#1$}}
{\mbox{\boldmath$\scriptscriptstyle\bf#1$}}}

% Random new commands
\newcommand{\bpi}{\bar \pi}
\newcommand{\setcond}[2]{\left\{\, #1 \,\st\, #2 \,\right\}}
\renewcommand{\P}{\mathcal{P}}
\newcommand{\D}{\mathcal{D}}
\renewcommand{\S}{\mathcal{S}}
\newcommand{\T}{\mathcal T}


\newcommand{\merge}{\mathbin{\lozenge}}
\newcommand{\mergeProj}{\mathbin{\lozenge_n^1}}

\newcommand{\bPizero}{\bar \Pi^E_{\mathrm{zero}(\T)}(\R^k, \Z^k)}
\newcommand{\bPizeroOne}{\bar \Pi^E_{\mathrm{zero}(\T)}(\R, \Z)}


%% Bold face letters
\newcommand{\rx}{{\ve r}}
\newcommand{\x}{{\ve x}}
\newcommand{\y}{{\ve y}}
\newcommand{\z}{{\ve z}}
\renewcommand{\v}{{\ve v}}
\newcommand{\g}{{\ve g}}
\newcommand{\e}{{\ve e}}
\renewcommand{\u}{{\ve u}}
\renewcommand{\a}{{\ve a}}
\newcommand{\f}{{\ve f}}
\newcommand{\0}{{\ve 0}}
\newcommand{\1}{{\ve 1}}
\newcommand{\m}{{\ve m}}
\newcommand{\p}{{\ve p}}
\renewcommand{\t}{{\ve t}}
\newcommand{\w}{{\ve w}}
\renewcommand{\b}{{\ve b}}
\renewcommand{\d}{{\ve d}}
\newcommand{\cve}{{\ve c}}
\newcommand{\h}{{\ve h}}
\newcommand{\rr}{{\ve r}}
\newcommand{\gp}{{\ve {\bar g}}}
\newcommand{\gt}{{\ve {\tilde g}}}
\newcommand{\gs}{{\ve  g}}

\newcommand{\ie}{ i.e. }

\newcommand{\Ball}{B}
% No longer mathcal
\newcommand{\B}{B}

\def\st{\mid}
\newenvironment{psmallmatrix}{\left(\smallmatrix}{\endsmallmatrix\right)}
\newenvironment{psmallmatrixbig}{\bigl(\smallmatrix}{\endsmallmatrix\bigr)}

%%% Redefine \pmod so that it respects \textstyle.
\makeatletter
\renewcommand{\pod}[1]%% a helper command used in \pmod.
{\allowbreak\mathchoice{\mkern18mu}{\mkern8mu}{\mkern8mu}{\mkern8mu}(#1)}
\makeatother

\chardef\Myunderscore=`\_
\newcommand\underscore{\Myunderscore\allowbreak}

\newcommand\githubsearchurl{https://github.com/mkoeppe/infinite-group-relaxation-code/search}
\input{sage-commands}
\DeclareRobustCommand\sage[1]{\texttt{#1}}
\DeclareRobustCommand\sagefunc[1]{\pgfkeys{/sagefunc/#1}}

\newcommand\NEWRESULT{$\clubsuit$}
\newcommand\TheoremNEWRESULT{New result \NEWRESULT}

\usepackage[normalem]{ulem}

 \newcommand{\tgreen}[1]{\textsf{\textcolor {ForestGreen} {#1}}}
 \newcommand{\tred}[1]{\texttt{\textcolor {red} {#1}}}
 \newcommand{\tblue}[1]{\textcolor {blue} {#1}}

\newif\ifJournalVersion\JournalVersiontrue

\JournalVersionfalse


\title[Infinite Group Problem Code: figures test suite]{Infinite Group Problem Code:\\figures test suite}

\begin{document}

\begin{abstract}
  The purpose of this file is to verify that changes to the code do not cause
  regressions in the plotting output.  In particular, we verify that the code
  samples in figure captions of published papers can create the same (or
  improved) figures.
  This file should be visually inspected periodically.
\end{abstract}

\maketitle

\section{Figures from \itshape Light on the Infinite Group Relaxation}

\begin{figure}[htbp]\centering
  \includegraphics[width=3.8cm]{graph-0}
  \includegraphics[width=3.8cm]{graph-b}
  \includegraphics[width=3.8cm]{graph-1}
  \includegraphics[width=3.8cm]{graph-2}
  \includegraphics[width=3.8cm]{graph-3}
  \includegraphics[width=3.8cm]{graph-12}
  \includegraphics[width=3.8cm]{graph-13}
  \includegraphics[width=3.8cm]{graph-23}
  %%%% This actually needs to go to the appropriate place in section 2!
  \caption{The hierarchy of valid, minimal, and extreme functions by
    example\dots
    Even without checking the
    dominance, it is easy to see that some functions cannot be
    minimal: they have some function values larger 
    than~$1$ (\emph{international orange}), but minimal valid functions are upper bounded by~1.
  }
  \label{fig:hierarchy}
\end{figure}

\begin{figure}
\begin{center}
%\includegraphics{flatFunction3and4.pdf}
\includegraphics[width=.31\linewidth]{not_extreme_1-covered_intervals}\quad
\includegraphics[width=.31\linewidth]{not_extreme_1-perturbation-1}\quad
\includegraphics[width=.31\linewidth]{not_extreme_1-perturbation-2}
\end{center}
\caption{This function (\sage{h = \sagefunc{not_extreme_1}()}) is
  minimal, but not extreme (and  hence also not a facet), as proved by
  \sage{\sagefunc{extremality_test}(h, show\underscore{}plots=True)}.
  The procedure first shows that
  for any distinct minimal $\pi^1 = \pi + \bar\pi$ (\emph{blue}), $\pi^2 = \pi
  - \bar\pi$ (\emph{red}) such that $\pi = \tfrac{1}{2}\pi^1
  + \tfrac{1}{2} \pi^2$, the functions $\pi^1$ and $\pi^2$ are continuous
  piecewise linear with the same breakpoints as $\pi$.  A finite-dimensional extremality
  test then finds two linearly independent perturbations $\bar\pi$ (\emph{magenta}), as shown.
  % , but the slopes on at
  % least some of the intervals will be different.
}
\label{fig:minimalNotExtreme}
\end{figure}

\begin{figure}[htbp]
  \centering
  %% Diagrams are a little bit to small if we try to squeeze 3 in a row.
  \includegraphics[width=.44\linewidth]{gj_forward_3_slope-2d_diagram}\quad
  \includegraphics[width=.44\linewidth]{not_minimal_2-2d_diagram}
  %\quad
  %\includegraphics[width=.31\linewidth]{not_extreme_1-2d_diagram}
  %\includegraphics[width=.4\linewidth]{the_irrational_function_t1_t2-2d_diagram}
  \caption{
    Two %Three 
    diagrams of a function (\emph{blue graphs on the top and the left}) and its polyhedral complex $\Delta\P$ (\emph{gray
      solid lines}), as plotted by the command
    \sage{\sagefunc{plot_2d_diagram}(h)}.
    \emph{Left}, \sage{h = \sagefunc{gj_forward_3_slope}()} \emph{(left)}.
    %\emph{Center}, 
    \emph{Right},
    \sage{h = \sagefunc{not_minimal_2}()}.
    %\emph{Right}, \sage{h = \sagefunc{not_extreme_1}()}.
    %Faces of the complex on which $\Delta\pi = 0$, i.e., additivity holds, are \emph{shaded green}.
    The set $E(\pi)$ in both cases is the union of the faces shaded in green.
    %\tred{(Refer to merit index somehow/somewhere. Use notation $E()$ somehow. --Matthias)}
    The \emph{heavy diagonal green line} $x + y = f$ % and $x + y = 1+f$ --
                                % not in these examples
    corresponds to the symmetry
    condition.  
    Vertices of $\Delta\P$ do not necessarily project (\emph{dotted gray lines}) to breakpoints.
    Vertices of the complex on which $\Delta\pi < 0$ are shown as \emph{red dots}.
    At the borders, the projections $p_i(F)$ of two-dimensional
    additive faces are shown as \emph{gray shadows}: $p_1(F)$ at the top border, $p_2(F)$ at
    the left border, $p_3(F)$ at the bottom and the right borders.
  }
  \label{fig:delta-p}
\end{figure}

\begin{figure}[htbp]
  \centering
  \includegraphics[width=.44\linewidth]{not_extreme_1-2d_diagram}
  \caption{
    %Two %Three 
    Diagram of a function (\emph{blue graphs on the top and the left}) on the
    evenly spaced complex~$\P_{\frac1{10}\Z}$ and the corresponding complex $\Delta\P_{\frac1{10}\Z}$ (\emph{gray
      solid lines}), as plotted by the command
    \sage{\sagefunc{plot_2d_diagram}(h)},
    where \sage{h = \sagefunc{not_extreme_1}()}.
    Faces of the complex on which $\Delta\pi = 0$, i.e., additivity holds, are \emph{shaded green}.
    The \emph{heavy diagonal green lines} $x + y = f$ and $x + y = 1+f$ correspond to the symmetry
    condition.  
    At the borders, the projections $p_i(F)$ of two-dimensional
    additive faces are shown as \emph{gray shadows}: $p_1(F)$ at the top border, $p_2(F)$ at
    the left border, $p_3(F)$ at the bottom and the right borders.
    Since the breakpoints of~$\P_{\frac1{10}\Z}$ are equally spaced, also $\Delta \P_{\frac1{10}\Z}$ is very
    uniform, consisting only of points, lines, and triangles, and the projections are
    either a breakpoint in $\P_{\frac1{10}\Z}$ or an interval in $\P_{\frac1{10}\Z}$; compare with
    \autoref{fig:delta-p}. 
  }
  \label{fig:uniform-spacing}
\end{figure}

\begin{figure}[htbp]

\centering
\includegraphics[width=.5\linewidth]{gmic-2d_diagram}
\caption{A diagram of a function of the type \sagefunc{gmic} (\emph{blue graphs on the top and the left}) and its polyhedral complex $\Delta\P$ (\emph{gray
      solid lines}), as plotted by the command
    \sage{\sagefunc{plot_2d_diagram}(gmic(f=2/3))}. There are three
    combinatorial types of these diagrams, depending on whether $f < \frac12$,
    $f = \frac12$, or $f > \frac12$.  No matter what $f$ is, the additivity domain
    $E(\pi)$ is the union of the faces $F_1 = F([0,f], [0,f], [0,f])$ and $F_2
    = F([f,1], [f, 1], [1 + f, 2])$, \emph{shaded in green}. 
    %% Vertices of the complex $\Delta\P$ do not necessarily project to
    %% breakpoints; compare with~\autoref{fig:uniform-spacing} [Part~I]. 
    At the borders of each diagram, the projections $p_i(F)$ of two-dimensional
    additive faces are shown as \emph{gray shadows}: $p_1(F)$ at the top border, $p_2(F)$ at
    the left border, $p_3(F)$ at the bottom and the right borders.}
\label{fig:gmic-additivity}
\end{figure}

\newcommand\LimitFigures[3]{
  \begin{minipage}[c]{0.3\linewidth}\includegraphics[width=\linewidth]{#1}\end{minipage}%
  \begin{minipage}[c]{0.3\linewidth}\includegraphics[width=\linewidth]{#2} \end{minipage}%
  \begin{minipage}[c]{0.07\linewidth}{\Huge$\ifJournalVersion\rightarrow\else\longrightarrow\fi$}\end{minipage}%
  \begin{minipage}[c]{0.3\linewidth}\includegraphics[width=\linewidth]{#3} \end{minipage}
}
\newcommand\LimitFiguresFour[4]{
  \begin{minipage}[c]{0.23\linewidth}\includegraphics[width=\linewidth]{#1}\end{minipage}%
  \begin{minipage}[c]{0.23\linewidth}\includegraphics[width=\linewidth]{#2} \end{minipage}%
  \begin{minipage}[c]{0.23\linewidth}\includegraphics[width=\linewidth]{#3} \end{minipage}%
  \begin{minipage}[c]{0.07\linewidth}{\Huge$\ifJournalVersion\rightarrow\else\longrightarrow\fi$}\end{minipage}%
  \begin{minipage}[c]{0.23\linewidth}\includegraphics[width=\linewidth]{#4} \end{minipage}
}
\begin{figure}[htbp]
  \centering
  \LimitFigures{drlm_gj_2_slope_extreme_limit_to_nonextreme_3-covered_intervals}{drlm_gj_2_slope_extreme_limit_to_nonextreme_50-covered_intervals}{drlm_gj_2_slope_extreme_limit_to_nonextreme-perturbation-1}
  \caption{A pointwise limit of extreme functions that is not extreme.  Consider the sequence of continuous extreme
    functions of type
    \sagefunc{gj_2_slope_repeat} set up for any $n \in \Z_+$ by
    \sage{h =
      \sagefunc{drlm_gj_2_slope_extreme_limit_to_nonextreme}(n)}.  For example,  $n = 3$ \emph{(left)} and $n=50$ \emph{(center)}.  This sequence 
    converges to a
    non-extreme discontinuous minimal valid function, set up with \sage{h =
      \sagefunc{drlm_gj_2_slope_extreme_limit_to_nonextreme}()}
    \emph{(right)}.  The limit function $\pi$ (\emph{black}) is shown with two
    minimal functions $\pi^1$ (\emph{blue}), $\pi^2$ (\emph{red}) such that
    $\pi = \frac12(\pi^1+\pi^2)$.} 
  \label{fig:drlm_gj_2_slope_extreme_limit_to_nonextreme}
\end{figure}

\begin{figure}[htbp]
  \centering
  \LimitFigures{bhk_irrational_extreme_limit_to_rational_nonextreme_1-covered_intervals}{bhk_irrational_extreme_limit_to_rational_nonextreme_2-covered_intervals}{bhk_irrational_extreme_limit_to_rational_nonextreme-perturbation-1} 
  \caption{A uniform limit of extreme functions that is not extreme.  
    The sequence of extreme functions of type \sagefunc{bhk_irrational}, set up with \sage{h =
      \sagefunc{bhk_irrational_extreme_limit_to_rational_nonextreme}(n)} where $n =
    1$ \emph{(left)}, $n=2$ \emph{(center)}, \dots\ converges to a non-extreme
    function, set up with \sage{h =
      \sagefunc{bhk_irrational_extreme_limit_to_rational_nonextreme}()}
    \emph{(right)}.
    The limit function $\pi$ (\emph{black}) is shown with two
    minimal functions $\pi^1$ (\emph{blue}), $\pi^2$ (\emph{red}) such that
    $\pi = \frac12(\pi^1+\pi^2)$ and a scaling of the perturbation
    function~$\bar\pi = \pi^1 - \pi$ (\emph{magenta}).} 
  \label{fig:bhk_irrational_extreme_limit_to_rational_nonextreme}
\end{figure}

\begin{figure}[tp]
  \centering
  \LimitFiguresFour{bccz_counterexample_with_ticks-0}{bccz_counterexample_with_ticks-1}{bccz_counterexample_with_ticks-2}{bccz_counterexample_with_ticks}
  % \scalebox{.4}{\includegraphics{limit_function.eps}} 
  \caption{First steps ($\psi_0 = \sage{\sagefunc{gmic}()}$, $\psi_1$, $\psi_2$) in the
    construction of the continuous non--piecewise linear limit function $\psi
    = \sage{\sagefunc{bccz_counterexample}()}$.}
  \label{fig:limit_function}
\end{figure}

\begin{figure}[tp]
\begin{center}
%\includegraphics{flatFunction3and4.pdf}
%% \includegraphics[width=.31\linewidth]{drlm_not_extreme_1-covered_intervals}\quad
\includegraphics[width=.31\linewidth]{drlm_not_extreme_1-perturbation-1}\quad
\includegraphics[width=.31\linewidth]{drlm_not_extreme_1-restricted-1}\quad
\includegraphics[width=.31\linewidth]{drlm_not_extreme_1-restricted-4-perturbation-1}
\end{center}
\caption{This function (\sage{h = \sagefunc{drlm_not_extreme_1}()}) is
  minimal, but not extreme (and  hence also not a facet), as proved by
  \sage{\sagefunc{extremality_test}(h, show\underscore{}plots=True)}
  by demonstrating a perturbation.  The red and blue perturbations describe the minimal functions $\pi^1, \pi^2$ that verify that $\pi$ is not extreme.  These minimal functions necessarily have more breakpoints than $\pi$.  This is because $\pi|_{\frac{1}{q} \Z}$ with $q = 7$, as depicted in the middle figure, is extreme for the finite group problem $R_f(\tfrac{1}{q} \Z,\Z)$.  However, $\pi|_{\frac{1}{2q} \Z}$ is not extreme for $R_f(\tfrac{1}{2q} \Z, \Z)$.  The discrete perturbations, depicted on the right, are interpolated to obtain the continuous functions $\pi^1, \pi^2$.
  % \tred{Refer to interpolation/restriction.}
  %\tred{NOTE this example only needs oversampling factor 2.}  \tred{Do we have an example of a function that is extreme on sampling factor 2 but not 3?}
}
\label{fig:drlm_not_extreme_1}
\end{figure}

\newcommand\CompendiumNEWRESULT{\hfill\par\vspace{1ex}\hfill \emph{Previously unpublished} \NEWRESULT}

\newcommand\CompendiumGraphics[1]{\vspace{0pt}\includegraphics[width=\linewidth]{#1}}
\newcommand\CompendiumGraphicsBig[1]{\multicolumn{2}{p{.25\linewidth}}{\vspace{0pt}\includegraphics[width=\linewidth]{#1}}}
\newcommand\CompendiumGraphicsTikz[1]{\vspace{0pt}\scalebox{.3}{\input{#1}}}

\newcommand\CompendiumSlopes[1]{\vspace{0pt}\centering#1}
\newcommand\CompendiumContinuity[1]{\vspace{0pt}\centering#1}

\newenvironment{CompendiumTabular}{%
    \begin{minipage}{\linewidth}\centering
      \let\footnoterule=\relax
    \begin{tabular}{p{.2\linewidth}*1{p{.17\linewidth}}@{}c*1{p{.05\linewidth}}*1{p{\ifJournalVersion.04\linewidth\else.09\linewidth\fi}}p{.35\linewidth}}
      \toprule
      Function\footnote{A function name shown in typewriter font is the name of the constructor of this function in the
        accompanying Sage program.}
      & Graph & %dummy column
      & Slopes & \ifJournalVersion Cont.\else Continuity\fi & Notes \\
      \midrule
      }
      {
      \bottomrule
    \end{tabular}
  \end{minipage}
}
\newenvironment{CompendiumProceduresTabular}{%
    \begin{minipage}{\linewidth}\centering
      \let\footnoterule=\relax
    \begin{tabular}{p{\ifJournalVersion.2\linewidth\else.3\linewidth\fi}*2{p{.17\linewidth}}p{.30\linewidth}}
      \toprule
      & \multicolumn{2}{c}{Graphs} \\
      \cmidrule{2-3}
      \multicolumn{1}{c}{Procedure\footnote{A procedure name shown in typewriter font is the name of the corresponding function in the
        accompanying Sage program.}}
      & \multicolumn{1}{c}{From} & \multicolumn{1}{c}{To}
      & \multicolumn{1}{c}{Notes} \\
      \midrule
      }
      {
      \bottomrule
    \end{tabular}
  \end{minipage}
}

\newcommand\sref[1]{{\bfseries \ParaSign\,\ref{#1}}}  %% Reference in section in
%% overview table

\renewcommand\multirowsetup{\relax}

\begin{table}[tp]
  \centering
  \caption{An updated compendium of known extreme functions for the
    infinite group problem V. Procedures.}
  \label{tab:compendium-procedures}

  \begin{CompendiumProceduresTabular}
    \vspace{0pt}
    \sagefunc{automorphism}
    & \CompendiumGraphics{automorphism-from}
    & \CompendiumGraphics{automorphism-to}
    & \vspace{0pt}
    From Johnson
    \\
    \vspace{0pt}
    \sagefunc{multiplicative_homomorphism}
    & \CompendiumGraphics{multiplicative_homomorphism-from}
    & \CompendiumGraphics{multiplicative_homomorphism-to}
    & \vspace{0pt}
    \\
    \vspace{0pt}
    \sagefunc{projected_sequential_merge}
    & \CompendiumGraphics{projected_sequential_merge-from}
    & \CompendiumGraphics{projected_sequential_merge-to}
    & \vspace{0pt}
    Operation $\mergeProj$ from Dey--Richard
    \\
    \vspace{0pt}
    \sagefunc{restrict_to_finite_group}
    & \CompendiumGraphics{restrict_to_finite_group-from}
    & \CompendiumGraphics{restrict_to_finite_group-to}
    & \vspace{0pt}
    Restrictions to finite group problems $R_f(\tfrac{1}{q} \Z, \Z)$ preserve
    extremality if $f$ and all breakpoints lie in $\tfrac{1}{q} \Z$. 
    \\
    \vspace{0pt}
    \sage{\sagefunc{restrict_to_finite_group}\allowbreak(oversampling=3)}
    & \CompendiumGraphics{restrict_to_finite_group_3-from}
    & \CompendiumGraphics{restrict_to_finite_group_3-to}
    & \vspace{0pt}
    If \sage{oversampling} by a factor $m \geq 3$, the restriction is extreme
    for $R_f(\tfrac{1}{mq} \Z, \Z)$ if and
    only if the original function is extreme.
    \\
    \vspace{0pt}
    \sagefunc{interpolate_to_infinite_group}
    & \CompendiumGraphics{interpolate_to_infinite_group-from}
    & \CompendiumGraphics{interpolate_to_infinite_group-to}
    & \vspace{0pt}
    Interpolation from finite group problems $R_f(\tfrac{1}{q} \Z, \Z)$
    preserves minimality, but in general not extremality. 
    \\
    \vspace{0pt}
    \sagefunc{two_slope_fill_in}
    & \CompendiumGraphics{two_slope_fill_in-from}
    & \CompendiumGraphics{two_slope_fill_in-to}
    & \vspace{0pt}
    Described by Gomory--Johnson, 
    Johnson.  For $k=1$, if % the
    % fill-in is 
    minimal, equal to
    \sagefunc{interpolate_to_infinite_group} (above).
    \\
  \end{CompendiumProceduresTabular}
\end{table}


\clearpage



\end{document}

%%% Local Variables:
%%% mode: latex
%%% TeX-master: t
%%% End:
